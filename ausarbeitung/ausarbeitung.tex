\documentclass[a4,german]{article}

\usepackage[german]{babel} % für deutsche Silbentrennung und generierte Texte
\usepackage[T1]{fontenc}    % für deutsche Umlaute etc. in der Ausgabe
\usepackage[utf8]{inputenc} % für deutsche Umlaute.
\usepackage{graphicx}       % um Bilder einzubinden
\usepackage{hyperref}       % um URLs korrekt einzubinden und Hyperlinks im Dokument zu ermöglichen


\begin{document}

\title{Irgendwas mit Wolken} % TODO
\author{Thilo Fryen, Lukas Hintze, Ali Pourasad, Maximilian Wetzel} % TODO: Alles richtig geschrieben?

\maketitle % setzt automatisch den angegebenen Autor und Titel, sowie das ehutige Datum als Überschirft (weitere Angaben möglich)


\begin{abstract}
    % TODO
    Wir klassifizieren Wolken.
    Und so.
\end{abstract}

\section{Einleitung}

% TODO: Einleitung
Hier sollte stehen, was under Thema ist, warum es interessant ist, was andere Leute davor schon gemacht haben,
was unsere Aufgabe ist, und wie unsere Arbeit aufgebaut ist.


\section{Daten}

% TODO: Daten
Hier sollte stehen, woher unsere Daten kommen, wie sie aussehen, was für Charakteristika sie haben, und was so klassentypische Einträge sind.


\section{Methodik}

% TODO: Methodik
Hier sollte stehen, wie wir unser Problem lösen, sprich, wie unser Endsystem funktioniert.
Warum machen wir es so und nicht anders?
Wie funktionieren die Verfahren, die wir nutzen?
Auch eine Ablaufgrafik wäre nice.


\section{Experimente bzw.\ Bewertung}

% TODO: Fazit
Wie sehen unsere Ergebnisse aus?
Wie verändern sich die Ergebnisse durch die CNNs?
Was passiert, wenn man einzelne Teile des Systems austauscht oder Merkmale entfern (dies evtl.\ auch mit Diagrammen zeigen)?
Wie schneidet das System pro Klasse ab?
Wie wirken sich Änderungen der Parameter/Hyperpara\-meter auf die Ergebnisse aus?
Wie schnell sind unsere Verfahren?


\section{Fazit}

% TODO: Fazit
Was ist unser Fazit?
Zusammenfassung und Einordnung unserer Ergebnisse.
Ausblick?


% Literaturverzeichnis mit Hilfe der "thebibliography" Umgebung:

\begin{thebibliography}{99}
	
% Beispiel für ein Buch
\bibitem {mustermann1234} M. Mustermann. \textit{Das hier ist nur ein Beispiel}. Musterverlag, 1234.

\end{thebibliography}



\end{document}
